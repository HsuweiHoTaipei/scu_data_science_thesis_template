\documentclass[12pt, a4paper, twoside, openright]{report}

% --------------------------------------------------
% Packages
% --------------------------------------------------
\usepackage{fontspec}
\usepackage{xeCJK}

% --- Watermark Control ---
\newif\ifwatermark
\watermarkfalse % Toggle: true = show watermark, false = hide

% --- Final Print Pages Control ---
\newif\iffinalprint
\finalprintfalse % Toggle: true to include signed pages (approval, authorization), false to skip

\usepackage{background}
\ifwatermark
    \backgroundsetup{
        contents={\includegraphics[width=6.25cm]{figures/watermark.png}},
        scale=1,
        opacity=1, 
        angle=0,
        position=current page.center,
        vshift=0pt,
        hshift=0pt
    }
\else
    \backgroundsetup{contents={}}
\fi

% 雙面列印邊界設定:內側 (inner) 較寬以利裝訂
% 您可以根據學校的具體規定微調這些數值
\usepackage[a4paper, twoside, inner=3.5cm, outer=2.5cm, top=2.5cm, bottom=2.5cm]{geometry}

\usepackage{amsmath}
\usepackage{amssymb}
\usepackage{graphicx}
\usepackage{booktabs}
\usepackage{indentfirst} % Indent first paragraph
\usepackage{emptypage}   % Clears headers/footers on blank pages
\usepackage{setspace}    % Line spacing
\usepackage{titlesec}    % Heading formatting
\usepackage{zhnumber}    % Chinese numbering
\usepackage{caption}     % Caption positioning
\usepackage{listings}    % Code blocks
\usepackage{algorithm}   % For algorithm blocks
\usepackage{algpseudocode} % For pseudocode inside algorithm
\usepackage[acronym]{glossaries} % For acronyms and glossaries
\usepackage{subcaption}  % For subfigures
\usepackage{xcolor}      % Colors for code
\definecolor{darkblue}{rgb}{0.0, 0.0, 0.55} % Define darker blue for printing

% Code Listing Settings
\lstset{
    basicstyle=\ttfamily\footnotesize,
    keywordstyle=\color{darkblue}\bfseries, % Use darkblue for keywords too
    commentstyle=\color{teal},
    stringstyle=\color{red},
    numbers=left,
    numberstyle=\tiny\color{gray},
    stepnumber=1,
    numbersep=5pt,
    backgroundcolor=\color{white},
    showspaces=false,
    showstringspaces=false,
    showtabs=false,
    frame=single,
    tabsize=2,
    breaklines=true,
    breakatwhitespace=false,
    captionpos=b,
    escapeinside={\%*}{*)}
}

% Fonts
\setCJKmainfont[
    AutoFakeBold=true,
    AutoFakeSlant=true
]{AR PL KaitiM Big5}
\setmainfont{TeX Gyre Termes}
\usepackage{tocloft}     % TOC formatting
\usepackage[natbibapa]{apacite} % APA Citation
\usepackage[colorlinks=true, linkcolor=darkblue, citecolor=darkblue, urlcolor=darkblue]{hyperref} % Clickable links with dark color
\usepackage[capitalise,noabbrev]{cleveref} % Smart cross-referencing, load AFTER hyperref
\usepackage{pdfpages}    % To include external PDF pages
\usepackage{fancyhdr}    % For customizing headers and footers
\usepackage{lipsum}      % English pseudo-text
% \usepackage{zhlipsum}    % Removed due to Simplified Chinese characters causing font warnings in Traditional Chinese template

% Manual Traditional Chinese Dummy Text
\newcommand{\lipsumTC}{
孔子曰:「大道之行也,天下為公。選賢與能,講信修睦。故人不獨親其親,不獨子其子;使老有所終,壯有所用,幼有所長,矜寡孤獨廢疾者,皆有所養。男有分,女有歸。貨惡其棄於地也,不必藏於己;力惡其不出於身也,不必為己。是故謀閉而不興,盜竊亂賊而不作,故外戶而不閉,是謂大同。」這是一段測試用的繁體中文假文,用以測試排版效果。此處文字將不斷重複以填滿版面。
孔子曰:「大道之行也,天下為公。選賢與能,講信修睦。故人不獨親其親,不獨子其子;使老有所終,壯有所用,幼有所長,矜寡孤獨廢疾者,皆有所養。男有分,女有歸。貨惡其棄於地也,不必藏於己;力惡其不出於身也,不必為己。是故謀閉而不興,盜竊亂賊而不作,故外戶而不閉,是謂大同。」這是一段測試用的繁體中文假文,用以測試排版效果。
}

% --------------------------------------------------
% Configuration
% --------------------------------------------------

% --- Glossary/Acronym Setup ---
\makenoidxglossaries % Generate glossaries without external script
% Define your acronyms here
\newacronym{nlp}{NLP}{Natural Language Processing}
\newacronym{cnn}{CNN}{Convolutional Neural Network}
\newacronym{svm}{SVM}{Support Vector Machine}

% --- Header/Footer Customization (using fancyhdr) ---
% 這會將所有頁碼統一置於頁尾中央,是常見的學術論文格式要求
\pagestyle{fancy}
\fancyhf{} % 清除所有頁眉與頁尾的預設值
\fancyfoot[C]{\thepage} % 將頁碼置於頁尾 (foot) 的中央 (C)
\renewcommand{\headrulewidth}{0pt} % 移除頁眉頂部的分隔線
\renewcommand{\footrulewidth}{0pt} % 移除頁尾底部的分隔線 (可選)

% 重新定義章節起始頁的 'plain' 樣式,使其與 'fancy' 樣式一致
\fancypagestyle{plain}{
  \fancyhf{}
  \fancyfoot[C]{\thepage}
  \renewcommand{\headrulewidth}{0pt}
}

% Line Spacing: 1.5
\linespread{1.5}
\setlength{\parindent}{2em} % 2 chars indentation

% Caption Positioning
\captionsetup[table]{position=top}
\captionsetup[figure]{position=bottom}

% Heading Formatting
% Chapter: 20pt, Centered, "第一章 OO"
% 使用 zhnumber 來自動產生中文章節編號
\titleformat{\chapter}[block]
  {\centering\fontsize{20}{30}\selectfont\bfseries}
  {第\zhnum{chapter}章\quad}
  {0pt}
  {}
\titlespacing*{\chapter}{0pt}{0pt}{20pt} % Adjust spacing as needed

% Section: 18pt, "第一節 OOO"
\titleformat{\section}
  {\fontsize{18}{27}\selectfont\bfseries}
  {第\zhnum{section}節\quad}
  {0pt}
  {}

% Subsection: 16pt, "一、OOO"
\titleformat{\subsection}
  {\fontsize{16}{24}\selectfont\bfseries}
  {\zhnum{subsection}、}
  {0pt}
  {}

% TOC Formatting to match "第一章..."
%\renewcommand{\cftchappresnum}{第}
%\renewcommand{\cftchapaftersnum}{章}
%\setlength{\cftchapnumwidth}{4em}

% Renaming
\renewcommand{\contentsname}{\centerline{目 錄}}
\renewcommand{\listfigurename}{\centerline{圖目錄}}
\renewcommand{\listtablename}{\centerline{表目錄}}
\renewcommand{\bibname}{\centerline{參考文獻}}
\renewcommand{\abstractname}{摘 要}

% --------------------------------------------------
% Document
% --------------------------------------------------
\begin{document}

% --- Cover Page ---
\begin{titlepage}
    % \NoBgThispage % Removed to fix conflict. Logic handled by manual check if needed, or accept watermark on title for now to debug.
    % Actually, better fix: disable background locally
    \backgroundsetup{contents={}} 
    \centering
    \vspace*{1cm}
    
    {\fontsize{24}{36}\selectfont 東吳大學巨量資料管理學院\par}
    {\fontsize{24}{36}\selectfont 資料科學系碩士班碩士論文\par}
    
    \vspace{2cm}
    
    {\fontsize{18}{27}\selectfont 指導教授:葉大雄 博士\par}
    % {\fontsize{18}{27}\selectfont \hspace{4.5em}○○○ 博士\par}
    
    \vspace{3cm}
    
    {\fontsize{24}{36}\selectfont 中文論文題目\par}
    {\fontsize{22}{33}\selectfont (English Thesis Title)\par}
    \vspace{1cm}
    
    \vfill
    
    {\fontsize{18}{27}\selectfont 研究生:小叮噹 撰\par}
    \newcounter{rocYear}
    \setcounter{rocYear}{\year}
    \addtocounter{rocYear}{-1911}
    \newcounter{rocMonth}
    \setcounter{rocMonth}{\month}
    {\fontsize{18}{27}\selectfont 中華民國 \zhnum{rocYear} 年 \zhnum{rocMonth} 月\par}
    
    \vspace*{1cm}
\end{titlepage}

% --- 審定頁與授權頁 (無頁碼) ---
% 使用 \cleardoublepage 確保每個獨立頁面都從右頁(奇數頁)開始
% \includepdf 會插入指定的 PDF 檔案,且預設不會有頁首、頁尾或頁碼
% 請將 'approval_page.pdf' 和 'ncl_authorization.pdf' 換成您自己的檔案路徑

% 當 \finalprinttrue 時,才會包含以下審定頁與授權頁
\iffinalprint
    \cleardoublepage
    \includepdf[pages=-, pagecommand={\backgroundsetup{contents={}}}]{approval_page.pdf} % 口試委員審定頁

    \cleardoublepage
    \includepdf[pages=-, pagecommand={\backgroundsetup{contents={}}}]{ncl_authorization.pdf} % 國家圖書館授權頁
\fi

% --- Front Matter (Roman Numbering) ---
\cleardoublepage % 確保摘要從右頁開始,且頁碼計算正確
\pagenumbering{roman}
\setcounter{page}{1}

% Chinese Abstract
\chapter*{摘 要}
\addcontentsline{toc}{chapter}{摘 要}
\lipsumTC

% English Abstract
\cleardoublepage
\chapter*{Abstract}
\addcontentsline{toc}{chapter}{Abstract}
\lipsum[1]

% Table of Contents
\cleardoublepage
\tableofcontents

% List of Tables
\cleardoublepage
\addcontentsline{toc}{chapter}{表目錄}
\listoftables

% List of Figures
\cleardoublepage
\addcontentsline{toc}{chapter}{圖目錄}
\listoffigures

% List of Acronyms
\cleardoublepage
\addcontentsline{toc}{chapter}{名詞縮寫對照表}
\printnoidxglossary[type=acronym, title={名詞縮寫對照表}]


% --- Main Matter (Arabic Numbering) ---
\cleardoublepage
\pagenumbering{arabic}
\setcounter{page}{1}

% Chapter 1
\chapter{緒論}
\section{研究背景}
本研究探討了機器學習在現代社會中的應用。根據 \citet{vaswani2017attention} 提出的 Transformer 架構,\gls{nlp} (Natural Language Processing) 取得了巨大突破。如 \cref{fig:main_fig_with_subs} 所示,我們可以比較不同模型的表現。\lipsumTC

\begin{table}[ht]
    \centering
    \caption{範例表格 (Example Table)}
    \label{tab:example}
    \begin{tabular}{c c}
        \toprule
        Item & Value \\
        \midrule
        A & 1 \\
        B & 2 \\
        \bottomrule
    \end{tabular}
\end{table}

\begin{figure}[ht] % A figure with two subfigures
    \centering
    \begin{subfigure}[b]{0.48\textwidth}
        \centering
        \includegraphics[width=\textwidth]{figures/watermark.png}
        \caption{子圖一 (Subfigure A)}
        \label{fig:sub1}
    \end{subfigure}
    \hfill % or \quad
    \begin{subfigure}[b]{0.48\textwidth}
        \centering
        \includegraphics[width=\textwidth]{figures/watermark.png}
        \caption{子圖二 (Subfigure B)}
        \label{fig:sub2}
    \end{subfigure}
    \caption{範例圖片(包含子圖)}
    \label{fig:main_fig_with_subs}
\end{figure}

\section{研究目的}
\citet{einstein} 的研究奠定了物理學基礎,而 \citet{knuth1984} 則為數位排版做出了貢獻。本計畫旨在整合這些理論。\lipsumTC

\section{研究範圍}
相關技術參考了 LaTeX 專案 \citep{latexproject}。\lipsumTC
如 \cref{tab:example} 所示,這是一個表格引用。

% Chapter 2
\chapter{文獻探討}
\section{相關理論}
\lipsumTC

% Chapter 3
\chapter{研究方法}
\section{資料收集}
\lipsumTC

\section{排版展示 (Formatting Demo)}
本節展示範本對各種格式的支援:

\subsection{字體效果 (Fonts)}
\begin{itemize}
    \item \textbf{粗體 (Bold)}:這是一段\textbf{粗體中文}與 \textbf{Bold English}。
    \item \textit{斜體 (Italic)}:這是一段\textit{斜體中文}與 \textit{Italic English}。
    \item \texttt{等寬字體 (Monospace)}:這是一段 \texttt{程式碼字體}。
\end{itemize}

\subsection{數學公式 (Mathematics)}
行內公式:$E = mc^2$。獨立公式如下:
\begin{equation}
    f(x) = \frac{1}{\sqrt{2\pi\sigma^2}} e^{-\frac{(x-\mu)^2}{2\sigma^2}}
\end{equation}

\subsection{程式碼 (Code)}
以下為 Python 程式碼範例:
\begin{lstlisting}[language=Python, caption={Python 範例代碼}]
def hello_world():
    # Print a message
    print("Hello, World!")
    return True

if __name__ == "__main__":
    hello_world()
\end{lstlisting}

\subsection{演算法實作 (Algorithm)}
本研究使用梯度下降法進行模型優化,其偽代碼如 \cref{alg:gd} 所示。
\begin{algorithm}[ht]
    \caption{梯度下降法 (Gradient Descent)}
    \label{alg:gd}
    \begin{algorithmic}[1]
        \Require 學習率 $\alpha$, 初始參數 $\theta$
        \Ensure 學習後的參數 $\theta$
        \While{$\theta$尚未收斂}
            \State 計算損失函數對 $\theta$ 的梯度: $J'(\theta)$
            \State 更新參數: $\theta \gets \theta - \alpha J'(\theta)$
        \EndWhile
        \State \Return $\theta$
    \end{algorithmic}
\end{algorithm}

% Chapter 4
\chapter{結果與討論}
\section{分析結果}
\lipsumTC

% Chapter 5
\chapter{結論與建議}
\section{結論}
\lipsumTC

% --- References ---
\cleardoublepage
\bibliographystyle{apacite}
\nocite{*} 
\bibliography{references}

% --- Appendix ---
\cleardoublepage
\appendix % 啟用附錄模式,後續的 chapter 會變成 "附錄 A", "附錄 B" ...

\chapter{附錄範例一}
\lipsumTC

\chapter{附錄範例二}
\lipsum[2]

\end{document}
